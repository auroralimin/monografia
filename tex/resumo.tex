Devido à crescente evolução na área de computação biomédica, escâneres de microscopia (estado da arte) tornaram-se capazes de capturar imagens de alta resolução de lâminas de tecido biológico em poucos minutos - chamadas de \textit{Whole-Slide tissue Images} (\textit{WSIs}). Entretanto, o processamento das imagens geradas por eles apresenta-se como um desafio, devido ao seu alto custo computacional. Nesse contexto atua o \textit{framework Region Templates}, projetado para lidar com \textit{pipelines} de processamento de imagens em ambientes distribuídos, híbridos e \textit{multi-core}. E, com o uso do paralelismo para a otimização do processamento de imagens de alta resolução, surge não só a necessidade de se quebrar essas imagens em pedaços ou \textit{tiles}, mas também de fazê-lo de modo otimizado dado o ambiente de execução. Desta forma, este trabalho tem como objetivo estender a implementação do \textit{Region Templates} e propor soluções, para particionamento de uma dada imagem, adaptáveis tanto à sua resolução e tamanho, quanto à quantidade e o tipo de núcleos de processamento disponíveis.
    